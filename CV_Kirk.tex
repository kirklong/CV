\documentclass[11pt]{article}
\usepackage{calc}
\usepackage{color}
\usepackage{enumitem}
\usepackage{url}
\urlstyle{same}
\usepackage[paper=a4paper,footskip=6mm,margin=14mm]{geometry}

\usepackage{fancyhdr,lastpage}
\fancyhf{}\renewcommand{\headrulewidth}{0pt}
\pagestyle{fancy}
\cfoot{\thepage\ of \pageref*{LastPage}}

\usepackage{titlesec}
\titleformat*{\section}{\large\bfseries}
\titlespacing\section{0pt}{12pt plus 6pt minus 4pt}{6pt plus 6pt minus 2pt}
\setcounter{secnumdepth}{0}

\usepackage[pdftitle={Kirk Long},pdfstartview=FitV,linktoc=section,bookmarks=true,colorlinks=false,linkcolor=blue,urlcolor=blue,anchorcolor=blue,citecolor=blue]{hyperref}

\setlength{\parindent}{0pt}
\newcommand{\forceindent}{\leavevmode{\parindent=2em\indent}}

%\newcommand{\highlight}[1]{\textbf{#1} }
\newcommand{\highlight}[1]{#1}

\newcommand{\makeheading}[2]%
        {\begin{minipage}[t]{\textwidth}%
                 {\LARGE \bfseries #1} \\[-0.3\baselineskip]%
                 \rule{\columnwidth}{1.5pt}\\[0.1\baselineskip]
         \end{minipage}}



\begin{document}

\makeheading{Kirk Long}{}

1440 N Locust Grove Rd				\hfill	+1 (208) 297-0396\\
BLDG 34 APT B	\hfill	\href{mailto:kirklong@u.boisestate.edu}{kirklong@u.boisestate.edu}\\
Meridian, ID, 83642			\hfill	\url{https://linkedin.com/in/kirkalan}



\section{Research Interests}
%%default: long version
 I am excited about employing advances in modern computing to analyze large data-sets and to simulate interesting systems numerically. I am broadly fascinated with the evolution of our universe and its systems, with a particular interest in dense stellar objects and the processes that form them (including the various flavors of neutron stars). I have enjoyed combining these passions in my research thus far, which has focused on identifying accreting pulsars in x-ray binaries.

\section{Education}
\begin{tabular}{ll}
08/2017 -- 05/2020	& 	\textbf{Boise State University}, Honors College \\
			&	Bachelors of Science in Physics, Astrophysics emphasis\\
			&	Minors in Music and Applied Mathematics\vspace{1mm} \\
			&	Cumulative GPA (including transfer credits): \textbf{3.859/4.0}\\
      & Physics GPA: \textbf{3.946/4.0} \\
      & Best GRE Scores: 166 (\textbf{89\textsuperscript{th} \%}) Quantitative | 164 (\textbf{94\textsuperscript{th} \%}) Verbal \\
      & \-\ \-\ \-\ \-\ \-\ \-\ \-\ \-\ \-\ \-\ \-\ \-\ \-\ \-\ \-\ \-\ \-\ \-\ \-\ \-\ \-\ \-\ \-\ \-\ \-\ 5.5 (\textbf{98\textsuperscript{th} \%}) Analytical Writing \vspace{2mm} \\
08/2015 -- 05/2017 & \textbf{Idaho State University} \\
      & Attended prior to transferring to Boise State, originally intended to major in math\\
      & and music here.\\

\end{tabular}

\section{Research Experience}
\begin{tabular}{ll}
04/2019 --		& 	\textbf{Identifying accreting x-ray binaries}, Boise State University \vspace{1mm} \\
			&	Mentored by Dr. Daryl Macomb, code sample available at: \url{https://github.com/kirklong}\\
      & $\bullet$ Analyzed archival data taken with both the CHANDRA and XMM-Newton \\
      &  \-\ \-\ \-\ orbital X-ray observatories, with data reduced via HEAsoft and SAS.\\
      & $\bullet$ Used FFT analysis to find periods of pulsar sources, then compared periods of \\
      & \-\ \-\ \-\ co-located sources in data from both instruments over time to detect changes \\
      & \-\ \-\ \-\ that could be driven by accretion. \\\
      & $\bullet$ Used techniques like bootstrapping to generate statistical significances of \\
      & \-\ \-\ \-\ potential detections from background data. \\
      & $\bullet$ Built folded light curves of potential new pulsar sources and accreting pairs. \\

\end{tabular}
\section{Teaching Experience}
\begin{tabular}{ll}
08/2018 --		& 	\textbf{Physics Lab Instructor}, Boise State University Department of Physics \vspace{1mm} \\
      & Responsible for teaching, grading, and managing class of students ($\sim$25/class).\\
      & Evaluations available for all courses upon request. Courses taught include: \vspace{2mm} \\
      & $\bullet$ PHYS 111 (General Physics I, Fa 2018)\\
      & $\bullet$ PHYS 105 (Introductory Stars and Cosmology, Sp 2019)\\
      & $\bullet$ PHYS 101 (Survey of Physics, Fa 2019)\\
      & $\bullet$ PHYS 204 (Introductory Astronomy, Fa 2019)\\
      & $\bullet$ PHYS 211 (General Physics I for Scientists and Engineers, Sp 2020) \\
      & \\
08/2018 --    &   \textbf{Physics Tutor}, Boise State University Department of Physics \vspace{1mm} \\
      & 1/2 students selected by Department to host drop-in tutoring lab (lead tutor 2019-2020).\\
      & Taught mostly lower-division coursework but occasionally helped with more advanced topics.\\
      & Average attendance per session: $\sim$5 students.\\

\end{tabular}
\newpage
\begin{tabular}{ll}
01/2019 --    &   \textbf{Volunteer}, Idaho Department of Corrections \\
      & Inspired by Bryan Stevenson's \textit{Just Mercy} to start program for inmates to learn STEM skills.\vspace{2mm} \\
      & Taught introductory programming (Python) class 1 hour/week with partial summer hiatus.\\
      & $\bullet$ Created curriculum based on programming classes taken and research experiences.\\
      & $\bullet$ Built Jupyter Notebook ``labs'' for inmates to follow. \\
      & \\
01/2020 --    &   \textbf{TA/Grader}, Boise State University Department of Physics
\vspace{1mm} \\
      & Responsible for grading homework assignments for upper division Classical Mechanics course. \\
\end{tabular}
\section {Outreach}
\begin{tabular}{ll}
06/2015 --    &   \textbf{Telescope Operator}, Bruneau Sand Dunes State Park Observatory \\
      & Former volunteer of $>$300 hours before being hired in March of 2017.\vspace{1mm} \\
      & Responsible for maintenance and operation of various large telescopes and \\
      & related equipment.\\
      & $\bullet$ Used telescopes to show visitors celestial objects with accompanying explanations\\
      & \-\ \-\ \-\ of both objects and equipment.\\
      & $\bullet$ Experience with various makes of telescopes with apertures up to 0.64 m (25").\vspace{1mm} \\
      & Tasked with creating and giving $\sim$45 minute public talks/presentations.\\
      & $\bullet$ Topics curated from latest research and most popular phenomena in astronomy, \\
      & \-\ \-\ \-\ distilled into form digestable by those without any previous background knowledge.\\
      & $\bullet$ Average crowd size: $\sim$150. Total visitors during employment: $>$20,000.\\
      & \\
01/2019 --    &   \textbf{Volunteer}, Idaho Department of Corrections  \\
      & Work detailed here in addition to volunteer teaching detailed above. \vspace{1mm} \\
      & Brought outreach events each week to local prisons to stimulate interest/continued \\
      & attendance in prison education programs (supplied by BSU Physics Department). \\
      & Occasionally helped with GED lessons when applicable to physics.\\
      & Originally volunteered at just men's facility but have recently (11/2019) started \\
      & expansion into women's as well.\vspace{1mm} \\
      & \textbf{Press:} Featured on Boise State University website (08/2019) and on local news \\
      & channel KIVI (11/2019). \\
      & \\
05/2016 --    &   \textbf{STEM Lead}, Treasure Valley Family YMCA \vspace{1mm} \\
      & In charge of writing STEM curricula for summer camps and after school programs \\
      & for all 4 local YMCA branches, including coding and physics programs. \vspace{0.5mm}\\
      & $\bullet$ Engaged and inspired $\sim$1,000 program attendees during employment. \\
      & $\bullet$ No child turned away: $\sim$1/5 of participants received financial assistance. \\
      & $\bullet$ Responsible for procuring/maintaining thousands of dollars worth of equipment. \\
      & $\bullet$ Trained $\sim$50 employees on STEM materials and how to effectively teach topics. \vspace{1mm} \\
      & Collected and analyzed budget and participant data throughout summer months. \\
      & \\
08/2016 -- 08/2018    &   \textbf{Intern}, \textit{StarTalk} \vspace{1mm} \\
      & Wrote blog posts on convoluted and/or newsworthy astronomy/physics topics. \\
      & $\bullet$ Posts disseminated to online audience of $>$500,000. \\
      & $\bullet$ Writing can be found by searching for my name on the \textit{StarTalk} \\
      & \-\ \-\ \-\ website: \url{https://www.startalkradio.net/?s=kirk+long} \\
      & $\bullet$ Occasionally still write guest posts, but main affilliation ended 2018. \\
      & \\
12/2019 --    &   \textbf{@ThreeBodyBot}, Twitter \#scicomm \vspace{1mm} \\
      & Built automated Twitter account that posts random three body simulations $\sim$1/day. \\
      & Source code available at \url{https://github.com/kirklong/ThreeBodyBot}. \\
\end{tabular}
\section{Scholarships and Awards}
\begin{tabular}{ll}
2015 -- 2017  &   Presidential Scholarship \hfill \$10,000\\
2017 -- 2018  &   Dean's Transfer Scholarship \hfill \$3,000\\
2018 -- 2019  &   Whitlock Math and Science Award \hfill \$800\\
2018 -- 2019  &   BSU Foundation Honors Award \hfill \$1,500\\
2018 -- 2019  &   Physics Department Scholarship \hfill \$1,000\\
2019 -- 2020  &   Honcik Physics Scholarship \hfill \$3,000\\
2019 -- 2020  &   George Campbell Memorial Award \-\ \-\ \-\ \-\ \-\ \-\ \-\ \-\ \-\ \-\ \-\ \-\ \-\ \-\ \-\ \-\ \-\ \-\ \-\ \-\ \-\ \-\ \-\ \-\ \-\ \-\ \-\ \-\ \-\ \-\ \-\ \-\ \-\ \-\ \-\ \-\ \-\ \-\ \-\ \-\ \-\ \-\ \-\ \-\ \-\ \-\ \-\ \-\ \-\ \-\ \-\ \-\ \-\ \-\ \-\ \-\ \-\ \-\ \-\ \-\ \-\ \-\ \-\ \-\ \-\ \-\ \hfill \$2,800\\
2016 --   &   Dean's List
\end{tabular}

\section{Skills (rated basic -- expert)}
\begin{itemize}[noitemsep]
\item \textbf{Programming Languages:}
    \begin{itemize}[noitemsep]
    \item Advanced: Python, Julia
    \item Competent: Matlab, Bash, C
    \item Basic: Perl, Fortran, JavaScript
    \end{itemize}
\item \textbf{Software:}
    \begin{itemize}[noitemsep]
    \item Expert: Microsoft Office Suite
    \item Advanced: Jupyter Lab/Notebook, Anaconda, terminal/command line
    \item Beginner: HEASOFT, SAS, LaTeX, MPI
    \item Basic: MESA, Git
    \end{itemize}
\item \textbf{Operating Systems:}
    \begin{itemize}[noitemsep]
    \item Advanced: Windows 10
    \item Competent: Mac OSX, Linux (Mint)
    \end{itemize}
\end{itemize}

\section{Posters and Publications}
\-\ \-\ Macomb, D. \& \textbf{Long, K.} (2020). Identifying accreting x-ray binaries. \textit{In preparation}.\vspace{1mm} \\
\-\ \-\ \textbf{Long, K.} (2018). \textit{To the Moon and Back -- Simulating the Trajectory of a Multi-Stage Rocket Similar to} \\
\-\ \-\ \textit{Saturn V in an Apollo 8 Mission Analogue.}\vspace{1mm}\\
\-\ \-\ Poster available on Github (\url{https://github.com/kirklong/Posters}), presented on:
\-\ \-\ \begin{itemize}[noitemsep]
              \item 02/2019: Research Computing Days, Boise State University
              \item 12/2018: Won best class poster from PHYS 325 (Scientific Computing)
        \end{itemize}
\-\ \-\ Barkley, K., Belnap, K., Keller, M., Larson, J., \textbf{Long, K.}, McCarthy, K., Myers, M., \& Withers, J. \\
 \-\ \-\ (2017). \textit{Idaho State University (a campus history)}. Charleston, SC: Arcadia. ISBN: 1467125512.

\section{Relevant Coursework}
\begin{tabular}{ll}
\textbf{Physics \& Math:} &   Classical Mechanics (\textbf{Taylor}), Electrostatics and Electrodynamics (\textbf{Griffiths}), \\
                  &   Introductory Quantum Mechanics (\textbf{Krane}), Introductory Relativity/Particle Physics, \\
                  &   Optics (\textbf{Pedrotti}), Thermal and Statistical Physics (\textbf{Schroeder}), Astrophysics,\\
                  &   Engineering Physics (\textbf{Knight}), Cosmology (Sp 20), Circuits (Sp 20),\\
                  &   Linear Algebra, Single/Multivariable Calculus (\textbf{Stewart}),\\
                  & Ordinary Differential Equations (\textbf{Boyce/DiPrima}) \vspace{1mm} \\
\textbf{Computing:} &   Scientific Computing (\textbf{Newman}), Computational Mathematics, \\
                    &   Parallel Scientific Computing (Sp 20), Introductory Computer Science\vspace{1mm}

%\textbf{Music:}   &   Music Theory I and II, Aural Skills I and II, Choir, Piano, Survey/History of Music \\
\end{tabular}
\section{Extracurricular Activities}
I enjoy spending time outdoors (particularly hiking and skiing), tinkering with amateur science projects (I've built cloud chambers, homemade rockets and fireworks, DIY telescopes, a brick kiln for metallurgical experimentation, and more), and making and teaching music (I've partly financed my studies thus far working at a local studio). I particularly enjoy learning and performing classical piano works---the latest addition to my repertoire is Gershwin's \textit{Rhapsody in Blue} and my favorite piece is Rachmaninoff's \textit{Prelude in G Minor}.

\end{document}
